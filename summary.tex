\documentclass[12pt]{article}





\usepackage{amsmath}
\usepackage{amsthm}
\usepackage{amssymb}
\usepackage{thmtools}

\newtheorem{theorem}{Theorem}


\theoremstyle{definition}
%\newtheorem{definition}{Definition}[]
\declaretheorem[style=definition,name=Definition,qed=$\lrcorner$]{definition}



\declaretheorem[style=definition,name=Example,qed=$\lozenge$]{example}


\theoremstyle{remark}
\newtheorem*{remark}{Remark}

\usepackage{todonotes}
\usepackage{mathpazo}
\usepackage{parskip}
\usepackage{eulervm}

\usepackage[a4paper, margin=1.2in]{geometry}
\usepackage{mdframed}

% begin custom commands
\newcommand{\Q}{\mathbb{Q}}
\newcommand{\R}{\mathbb{R}}
\newcommand{\N}{\mathbb{N}}
\newcommand{\F}{\mathcal{F}}
\newcommand{\X}{\mathcal{X}}
\newcommand{\W}{\mathcal{W}}
\newcommand{\Var}{\operatorname{Var}}
\newcommand{\E}{\mathbb{E}}






\title{\textsc{\large Lecture Summary}\\Economics and Computation }
\author{David Zollikofer}


\begin{document}

	\maketitle
	
\section*{2. Simultaneous Move Games}
This chapter introduces a formal framework with which we can analyze decision making. 

\begin{remark}[Rationality]
In the discussion we will always assume the rationality of all involved parties. This means that they will always choose a higher payoff to a lower one. The way actions are mapped to payoffs (called the utility function) is completely individual.
\end{remark}

\begin{definition}[Preference order]
	For a set of actions $A = (a_1,\ldots, a_n)$ we require a preference ordering on $A$ to fulfill:
	\begin{itemize}
		\item \textit{Completeness:} any two actions $a_1$, $a_2$ we either have $a_1 \succeq a_1$ or $a_1 \preceq a_2$.
		\item \textit{Transitivity:} if $a_1 \preceq a_2$ and $a_2 \preceq a_3$ then $a_1 \preceq a_3$.
	\end{itemize}
	We call $\succeq$ to be weakly better and $\succ$ to be strictly better.
\end{definition}
It is important to note that it is unclear how much better an action is than another. For this we need the utility function.
\begin{definition}[Utility function]
	A utility function $u:A\to \R$ maps an action $a\in A$ to a payoff of that action. If for a player $a_i \succeq a_j$, then $u(a_i) \geq u(a_j)$ must hold.
\end{definition}

\begin{definition}[Simultaneous-move games]
A simultaneous move game is a triple $(N,A,u)$ consisting of 
\begin{itemize}
	\item $N = \{1,\ldots,n\}$ is the number of players.
	\item Each player plays action $a_i \in A_i$ with $A = A_1 \times \ldots \times A_n$. An action profile is a vector $\vec{a} = (a_1,\ldots,a_n)\in A$
	\item The utility vector $\vec{u} = (u_1,\ldots,u_n)$ consists of the player's utility functions and maps an action profile to a payoff.
\end{itemize}
Simultaneous means that I do not know what someone else has decided when I take a decision.
\end{definition}

\begin{definition}[Pareto dominated]
	We call an action profile $a\in A$ Pareto dominated by $a'\in A$ if (looking at the utility of both profiles) no player is worse off in $a$ than in $a'$ and at least one player is better of in $a'$ than $a$.
\end{definition}

\begin{definition}[Pareto optimality]
	An action profile $a\in A$ is Pareto optimal iff. there is not other action profile that Pareto dominates $a$. This means there is no action profile that makes someone better off without also making someone worse off.
\end{definition}

Similarly we can also introduce pareto optimality for distributions on action profiles.

\begin{definition}[Dominant-strategy equilibrium]
	We call an action profile $a^*$ a dominant-strategy equilibrium of a simultaneous move game iff. for any other action any other player could take 
	$$u_i(a^*_i,a') \geq u_i(j_i,a')$$
	\todo{finish definition}
\end{definition}

\begin{definition}[Pure-strategy Nash equilibrium]
	content...\todo{add}
\end{definition}

\begin{definition}[Mixed strategy]
	content...\todo{text}
\end{definition}

\begin{definition}[Mixed-strategy Nash equilibrium]
	content... \todo{text}
\end{definition}

\begin{definition}[Support]
The support of a mixed-strategy Nash equilibrium is the set of actions that are played with strictly positive probability.
\end{definition}


\begin{theorem}[Existence of mixed-strategy Nash equilibrium]
	Every finite simultaneous-move game has at least one mixed-strategy Nash equilibrium.
\end{theorem}


\todo{a lot missing here}

\begin{definition}[Zero-sum game]
	Wel call a game a zero-sum game if the sum of all payoffs for every action profile is 0.
\end{definition}

\todo{add rest}


\section*{5. Peer-to-Peer Systems}

\begin{definition}[Peer-to-Peer System] 
In a peer-to-peer file sharing system the seperation between servers and clients is removed. Every participant is referred to as a peer. The distributed nature often eliminates a single point of failure while achieving a similar degree of reliability as traditional client server models.

A peer-to-peer system is characterized by 
\begin{itemize}
	\item \textit{Protocol:} The protocol defines the actions that are supported and correspond to rules in game theory.
	\item \textit{Reference client:} A client that implements a default strategy in game theory.
	\item \textit{Other clients:} Other clients can implement other game theoretical strategies, that conform to the protocol.
\end{itemize}

When analyzing peer-to-peer networks we are interested in the following metrics:

\begin{itemize}
	\item \textit{Performance:} e.g. average download speed
	\item \textit{Incentive Properties:} e.g. sharing content should be incentivized.
	\item \textit{Fairness:} e.g. my uploads should be proportional to my downloads.
\end{itemize}

\end{definition}

\begin{example}[Napster]
A central server contained a mapping from content to providers. To download I file I would query the central server for a peer that contains the file and then download it directly from the peer.
\end{example}

\begin{example}[Gnutella]
There are central servers managing a list of peers IP addresses. To find content a peer $P_A$ send a query message which gets passed from peer to peer until a peer $P_B$ has the desired file (or there is a timeout). The file is then downloaded directly from peer $P_B$ to peer $P_A$. The protocol does not keep any statistics on who downloaded from whom and who seeded.
\end{example}

\begin{remark}[Free Riding]
The Gnutella protocol has the problem that the incentives are isomorphic to the prisoner's dilemma; in that every peer tries to \textbf{free ride} which is to download but not seed.
\end{remark}


\begin{example}[BitTorrent]
	
\begin{itemize}
	\item \textit{Torrent:} A \texttt{.torrent} file consists of a list of all blocks of which file we would like to download consists as well as their respected hashes to verify their integrity as well as a tracker URL. The tracker is a server which coordinates peers using the same torrent.
	
	\item \textit{Swarm:} The set of peers either downloading or uploading the same file. The tracker provides you with a random number of peers that are currently active in the swarm.
	
	\item \textit{Seeders:} Peers who have already downloaded the file and are sharing it.
	
	\item \textit{Leechers:} Peers who are downloading pieces of the file.
\end{itemize}
\end{example}





\section*{6. Auction Design}

\subsection*{Sealed-Bid Auctions}

\begin{definition}[Sealed-Bid Auction]
We have the set of bidders $N = \{0,\ldots n\}$ and every bidder values an item at $v_i$ In a sealed bid aution every bidder submits a bid $b_i$ which creates a bid profile $\vec{b}  = (b_1,\ldots, b_n)$ where the bidders do not know the bids of the other bidders.

For a given bid profile $\vec{b}$ we have an allocatition rule $x(\vec{b})$ which returns a one-hot vector which encodes which bidder won the auction. Furthermore the payment rule $t(\vec{b})$ is a one-hot vector which contains the payment (bid) by the winning bidder.

We know the following two sealed-bid auctions:

\begin{itemize}
	\item \textit{First-price sealed-bid auction (FPSB)}: The highest bidding bidder  wins and pays its bid.
	\item \textit{Second-price sealed-bid auction (SPSB) aka. Vickrey Auction} The highest bidding bidder wins but only pays the bid of the second highest bidder.
\end{itemize}


\end{definition}

\begin{definition}[Values] In auctions we differentiate between the following different values bidders can have: 
	\begin{itemize}
		\item \textit{Private value}: A bidder knows her own value (e.g. ebay book auction) and the value doesn't change if she knows the values of other bidders.
		\item \textit{Interdependent value}: All bidders have different values despite having the same information available, but they might change their value if they learn the value of other bidders (e.g. copper mining rights but different valuations of the rights exist).
		\item \textit{Common Value}: All bidders may be uncertain about their value but they value the item the same and have the same information; yet they might revise their bids if they learn about the values of others (e.g. art auction).
	\end{itemize}
\end{definition}


\begin{definition}[Quasi-linear utility]
	Given bid profile $b$, the utility of bidder $i$ is defined as
	\begin{align*}
		u_i(b) &= x_i(b) v_i - t_i(b)
	\end{align*}
	where $u_i$ is the utility of bidder $i$, $t_i$ the payment made by bidder $i$, $v_i$ the valuation of the item for bidder $i$ and $x_i$ an indicator variable whether bidder $i$ got the item or not.
\end{definition}


\begin{definition}[Strategy]
	A strategy maps a value to a bid (e.g. Given I value the shoes 10 CHF how much will I bid). For a value profile $\vec{v} = (v_1,\ldots,v_n)$ for all players we have the strategy profile $\vec{b} = (b_1,\ldots,b_n) = s(\vec{v})$.
\end{definition}


\begin{definition}[Dominant-Strategy Equilibrium (DSE)]
	\todo{when does it exist?}
We call $s^* = (s_1^*, \ldots, s_n^*)$ a DSE iff. for all bidders $i$, all values $v_i$, all bids $b_i$ and all other players values $v_{i-}$ and bids $b_{i-}$ we have
\begin{align*}
	u_i(s_i^*(v_i),s_{i-}(v_{i-})) &\geq 	u_i(b_i,s_{i-}(v_{i-}))
\end{align*}
In words we want that the strategy optimizes every bidders value independent of the bids and values of the other players.
\end{definition}



\begin{definition}[Strategy-proof]
We call an auction strategy-proof if the truthful bidding is a dominant-strategy equilibrium.
\end{definition}


\begin{theorem}
	Second-price sealed-bid actions are strategy-proof and efficient.
\end{theorem}



\begin{definition}[Bayes-Nash Equilibrium]
	A strategy profile $s^* = (s_1^*,\ldots,s_n^*)$ is a Bayes-Nash equilibrium (BNE) in a sealed bid auction iff. for all bidders $i$, all values $v_i$ and all bids $b_i$ we have
	\begin{align*}
\E_{v-i}\left[u_i(s_i^*(v_i),s_{-i}^*(v_{-i}))\right] &\geq \E_{v-i}\left[u_i(b_i,s_{-i}^*(v_{-i}))\right]
	\end{align*}
	We have assumed that the bidder's values follow $v_i\sim G_i$ which is a probability distribution function. The $G_i$ are common knowledge to all biders and set up the so called independent private value (IPV). \footnote{if the $G_i$ are indepdendent and identically distributed this is known as the IID private value environment.}
\end{definition}




\begin{theorem}
	For bidders with IID values uniformly distributed on $[0,1]$, the Bayes-Nash Equilibrium in the FPSB has strategy
	\begin{align*}
		s^*(v_i) = \left(\frac{n-1}{n}\right)v_i
	\end{align*}
	This makes the FPSB auction efficient with this equilibrium.
\end{theorem}

\begin{proof}[Proof sketch]
	\todo{why not of other form?}
We assume $n=2$ bidders and that $s_i(v_i) = \alpha v_i$. For bidder 1 the payoff is $$(v_1 - b_1) \Pr \left[b_2\leq b_1\right] = (v_1 - b_1 ) \frac{b_1}{\alpha}.$$ Optimizing the expression with respect to $\beta_1$ yields the desired result.
\end{proof}

\begin{remark}
	This makes the FPSB auction non strategy-proof.
\end{remark}

\subsection*{Multi-round Auctions}

There are five main types of multi-round auctions:

\begin{itemize}
	content...
\end{itemize}



\end{document}