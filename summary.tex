\documentclass[12pt]{article}


\usepackage{lipsum}


\usepackage{amsmath}
\usepackage{amsthm}
\usepackage{amssymb}
\usepackage{thmtools}

\newtheorem{theorem}{Theorem}


\theoremstyle{definition}
%\newtheorem{definition}{Definition}[]
\declaretheorem[style=definition,name=Definition,qed=$\lrcorner$]{definition}



\declaretheorem[style=definition,name=Example,qed=$\lozenge$]{example}


\theoremstyle{remark}
\newtheorem*{remark}{Remark}

\usepackage{todonotes}
\usepackage{mathpazo}
\usepackage{parskip}
\usepackage{eulervm}

\usepackage[a4paper, margin=1.2in]{geometry}
\usepackage{mdframed}

% begin custom commands
\newcommand{\Q}{\mathbb{Q}}
\newcommand{\R}{\mathbb{R}}
\newcommand{\N}{\mathbb{N}}
\newcommand{\F}{\mathcal{F}}
\newcommand{\X}{\mathcal{X}}
\newcommand{\W}{\mathcal{W}}
\newcommand{\Var}{\operatorname{Var}}
\newcommand{\E}{\operatorname{E}}




\usepackage{mathpazo}

\title{\textsc{\large Lecture Summary}\\Economics and Computation }
\author{David Zollikofer}


\begin{document}

	\maketitle
	
\section*{2. Simultaneous Move Games}
This chapter introduces a formal framework with which we can analyze decision making. 

\begin{remark}[Rationality]
In the discussion we will always assume the rationality of all involved parties. This means that they will always choose a higher payoff to a lower one. The way actions are mapped to payoffs (called the utility function) is completely individual.
\end{remark}

\begin{definition}[Preference order]
	For a set of actions $A = (a_1,\ldots, a_n)$ we require a preference ordering on $A$ to fulfill:
	\begin{itemize}
		\item \textit{Completeness:} any two actions $a_1$, $a_2$ we either have $a_1 \succeq a_1$ or $a_1 \preceq a_2$.
		\item \textit{Transitivity:} if $a_1 \preceq a_2$ and $a_2 \preceq a_3$ then $a_1 \preceq a_3$.
	\end{itemize}
	We call $\succeq$ to be weakly better and $\succ$ to be strictly better.
\end{definition}
It is important to note that it is unclear how much better an action is than another. For this we need the utility function.
\begin{definition}[Utility function]
	A utility function $u:A\to \R$ maps an action $a\in A$ to a payoff of that action. If for a player $a_i \succeq a_j$, then $u(a_i) \geq u(a_j)$ must hold.
\end{definition}

\begin{definition}[Simultaneous-move games]
A simultaneous move game is a triple $(N,A,u)$ consisting of 
\begin{itemize}
	\item $N = \{1,\ldots,n\}$ is the number of players.
	\item Each player plays action $a_i \in A_i$ with $A = A_1 \times \ldots \times A_n$. An action profile is a vector $\vec{a} = (a_1,\ldots,a_n)\in A$
	\item The utility vector $\vec{u} = (u_1,\ldots,u_n)$ consists of the player's utility functions and maps an action profile to a payoff.
\end{itemize}
Simultaneous means that I do not know what someone else has decided when I take a decision.
\end{definition}

\begin{definition}[Pareto dominated]
	We call an action profile $a\in A$ Pareto dominated by $a'\in A$ if (looking at the utility of both profiles) no player is worse off in $a$ than in $a'$ and at least one player is better of in $a'$ than $a$.
\end{definition}

\begin{definition}[Pareto optimality]
	An action profile $a\in A$ is Pareto optimal iff. there is not other action profile that Pareto dominates $a$. This means there is no action profile that makes someone better off without also making someone worse off.
\end{definition}

Similarly we can also introduce pareto optimality for distributions on action profiles.

\begin{definition}[Dominant-strategy equilibrium]
	We call an action profile $a^*$ a dominant-strategy equilibrium of a simultaneous move game iff. for any other action any other player could take 
	$$u_i(a^*_i,a') \geq u_i(j_i,a')$$
	\todo{finish definition}
\end{definition}

\begin{definition}[Pure-strategy Nash equilibrium]
	content...\todo{add}
\end{definition}

\begin{definition}[Mixed strategy]
	content...\todo{text}
\end{definition}

\begin{definition}[Mixed-strategy Nash equilibrium]
	content... \todo{text}
\end{definition}

\begin{definition}[Support]
The support of a mixed-strategy Nash equilibrium is the set of actions that are played with strictly positive probability.
\end{definition}


\begin{theorem}[Existence of mixed-strategy Nash equilibrium]
	Every finite simultaneous-move game has at least one mixed-strategy Nash equilibrium.
\end{theorem}


\todo{a lot missing here}

\begin{definition}[Zero-sum game]
	Wel call a game a zero-sum game if the sum of all payoffs for every action profile is 0.
\end{definition}

\todo{add rest}


\section*{5. Peer-to-Peer Systems}

\begin{definition}[Peer-to-Peer System] 
In a peer-to-peer file sharing system the seperation between servers and clients is removed. Every participant is referred to as a peer. The distributed nature often eliminates a single point of failure while achieving a similar degree of reliability as traditional client server models.

A peer-to-peer system is characterized by 
\begin{itemize}
	\item \textit{Protocol:} The protocol defines the actions that are supported and correspond to rules in game theory.
	\item \textit{Reference client:} A client that implements a default strategy in game theory.
	\item \textit{Other clients:} Other clients can implement other game theoretical strategies, that conform to the protocol.
\end{itemize}

When analyzing peer-to-peer networks we are interested in the following metrics:

\begin{itemize}
	\item \textit{Performance:} e.g. average download speed
	\item \textit{Incentive Properties:} e.g. sharing content should be incentivized.
	\item \textit{Fairness:} e.g. my uploads should be proportional to my downloads.
\end{itemize}

\end{definition}

\begin{example}[Napster]
A central server contained a mapping from content to providers. To download I file I would query the central server for a peer that contains the file and then download it directly from the peer.
\end{example}

\begin{example}[Gnutella]
There are central servers managing a list of peers IP addresses. To find content a peer $P_A$ send a query message which gets passed from peer to peer until a peer $P_B$ has the desired file (or there is a timeout). The file is then downloaded directly from peer $P_B$ to peer $P_A$. The protocol does not keep any statistics on who downloaded from whom and who seeded.
\end{example}

\begin{remark}[Free Riding]
The Gnutella protocol has the problem that the incentives are isomorphic to the prisoner's dilemma; in that every peer tries to \textbf{free ride} which is to download but not seed.
\end{remark}


\begin{example}[BitTorrent]
	
\begin{itemize}
	\item \textit{Torrent:} A \texttt{.torrent} file consists of a list of all blocks of which file we would like to download consists as well as their respected hashes to verify their integrity as well as a tracker URL. The tracker is a server which coordinates peers using the same torrent.
	
	\item \textit{Swarm:} The set of peers either downloading or uploading the same file. The tracker provides you with a random number of peers that are currently active in the swarm.
	
	\item \textit{Seeders:} Peers who have already downloaded the file and are sharing it.
	
	\item \textit{Leechers:} Peers who are downloading pieces of the file.
\end{itemize}
\end{example}


\end{document}